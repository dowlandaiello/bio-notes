\documentclass{article}

\usepackage[english]{babel}
\usepackage{microtype}
\usepackage{graphicx}
\usepackage{wrapfig}
\usepackage{enumitem}
\usepackage{fancyhdr}
\usepackage{amsmath}
\usepackage{chemformula}
\usepackage{index}
\usepackage{hyperref}
\usepackage[margin=1.0in]{geometry}
\usepackage{qtree}
\usepackage{float}

\begin{document}
\title{Summary: The Nature of Chemical Regulation}
\author{Dowland Aiello}
\date{April 1, 2020}

\maketitle
\tableofcontents
\fancyhf{}

\newpage

\section{Communication through electrical and chemical signals}

The \textbf{nervous system} and the \textbf{endocrine system} serve as the
body's principal coordination organ systems. As their names suggest, the
endocrine system establishes coordination in a \emph{chemical} manner, while the
nervous system establishes such coordination in an \emph{electrical} manner.

\subsection{An overview of the endocrine system}

The endocrine system establishes communication by releasing chemical signals
called \textbf{hormones}.

\end{document}
