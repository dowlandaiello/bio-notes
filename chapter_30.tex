\documentclass{article}

\usepackage[english]{babel}
\usepackage{microtype}
\usepackage{graphicx}
\usepackage{wrapfig}
\usepackage{enumitem}
\usepackage{fancyhdr}
\usepackage{amsmath}
\usepackage{chemformula}
\usepackage{index}
\usepackage{hyperref}
\usepackage[margin=1.0in]{geometry}
\usepackage{qtree}
\usepackage{float}
\usepackage{booktabs}
\usepackage{tabularx}
\usepackage{textcomp}

\begin{document}
\title{Summary: Locomotion}
\author{Dowland Aiello}
\date{April 20, 2020}

\maketitle
\tableofcontents
\fancyhf{}

\newpage

\section{Overview: Locomotion}

When an animal isn't moving, it is kept in place through two forces: gravity and
friction. When an organism overcomes these forces and moves, an organism
engages in \textbf{locomotion}, or active travel from one place to another.
Regardless of implementation details that differ between organisms, all types of
locomotion require the movement of protein strands against each other. This
behavior is demonstrated in avrious organisms and environments, such as:

\begin{itemize}
    \item \textbf{Acquatic} animals: due the density of the water encompassing their
    natural environment, these organisms need not hastily concern themselves with
    overcoming the force of gravity, but, rather, friction.
    \item \textbf{Terrestrial} animals: due to the lack of support provided by air,
    terrestrial organisms must expend energy to counter the force of gravity,
    and, additionally, friction with respect to the ground.
    \begin{itemize}
        \item Walking animals
            \begin{itemize}
                \item Bipedal animals: utilize just two limbs (legs) in
                maintaining balance, maintaining contact with the ground on at
                least one foot, but lack the stability of their quadrapedal or
                tripedal equivalents
                \item Quadrapedal animals: constantly utilize at least three feet
                in maintaining balance
                \item In both quadrapedal and bipedal animals, momentum plays
                an important part in the persistence of balance while running
            \end{itemize}
        \item Hopping animals
            \begin{itemize}
              \item Kangaroos: travel by generating power in hind leg muscles,
                and by storing energy in connected tendons
            \end{itemize}
        \item Crawling animals
            \begin{itemize}
                \item Energy expenditure is focused towards friction
                \item Snakes: entire body is undulated from side to side
                \item Earthworms: perform a sequence of muscle contractions from
                head to tail (\textbf{peristalsis})
            \end{itemize}
        \item Flying animals
            \begin{itemize}
                \item Energy expenditure is cofused towards combating gravity
                and developing ``lift.''
            \end{itemize}
    \end{itemize}
\end{itemize}

\end{document}
