\documentclass{article}

\usepackage[english]{babel}
\usepackage{microtype}
\usepackage{graphicx}
\usepackage{wrapfig}
\usepackage{enumitem}
\usepackage{fancyhdr}
\usepackage{amsmath}
\usepackage{chemformula}
\usepackage{index}
\usepackage{hyperref}
\usepackage[margin=1.0in]{geometry}
\usepackage{qtree}
\usepackage{float}
\usepackage{booktabs}
\usepackage{tabularx}
\usepackage{textcomp}

\begin{document}
\title{Summary: Locomotion}
\author{Dowland Aiello}
\date{April 20, 2020}

\maketitle
\tableofcontents
\fancyhf{}

\newpage

\section{Overview: Locomotion}

When an animal isn't moving, it is kept in place through two forces: gravity and
friction. When an organism overcomes these forces and moves, an organism
engages in \textbf{locomotion}, or active travel from one place to another.
Regardless of implementation details that differ between organisms, all types of
locomotion require the movement of protein strands against each other. This
behavior is demonstrated in avrious organisms and environments, such as:

\begin{itemize}
    \textbf{Acquatic} animals: due the density of the water encompassing their
    natural environment, these organisms need not hastily concern themselves with
    overcoming the force of gravity, but, rather, friction.
    \textbf{Terrestrial} animals: due to the lack of support provided by air,
    terrestrial organisms must expend energy to counter the force of gravity,
    and, additionally, friction with respect to the ground.
\end{itemize}

\end{document}
