\documentclass{article}

\usepackage[english]{babel}
\usepackage{microtype}
\usepackage{graphicx}
\usepackage{wrapfig}
\usepackage{enumitem}
\usepackage{fancyhdr}
\usepackage{amsmath}
\usepackage{chemformula}
\usepackage{index}
\usepackage{hyperref}
\usepackage[margin=1.0in]{geometry}
\usepackage{qtree}
\usepackage{float}
\usepackage{booktabs}
\usepackage{tabularx}
\usepackage{textcomp}

\begin{document}
\title{Summary: Community Structure and Dynamics}
\author{Dowland Aiello}
\date{May 28, 2020}

\maketitle
\tableofcontents
\fancyhf{}

\newpage

\section{Overview: The Community as a Hierarchy}

In order to better analyze the various actors involved in some process, one might find that grouping various
actors involved at a certain geographical location or in respect to some inquiry can provide utility and make
answering such an inquiry easier. In doing so, one must analyze the various \textbf{populations} or groups of
interacting organisms of the same species, and, on a larger scale, the biological \textbf{communities} or groups
of geographically proximate populations that one deems paramount to successfully addressing the matter of inquiry.

In addition to the above classifications, a researcher may choose to bear regards to the composition of a community
--the feeding relationships, species, and number of populations within the community--, as well as the interactions
between the populations, which each may differ in their importance and scope. Over longer periods of time, these
interactions may manifest themselves in the study of community dynamics, which deals with the analysis of patterns
in the populatiion makeup of a community.

\section{Interactions in a Community}

Among the members of a community, \textbf{interspecific interactions} can be classified according to their effects
on the two populations involved. Interspecific \emph{competition}, for example, is generally a mutually harmlful
act, whereas \textbf{mutualism} (e.g., dichotomy of pollinators and flowers) is, in most cases, mutually 
beneficial. A third class of interspecific relations, by contrast, bears benefit to only one of the involved
parties. Examples of such a relationship are:

\begin{itemize}
	\item \textbf{Predation}: predator benefits, while the prey does not (+/-). Encourages adaptation in the
	prey population.
	\item \textbf{Herbivory}: consumer benefits, while the plant or algae does not (+/-). Encourages
	development of defense mechanisms in plants (e.g., thorns and spines), as well as chemical defense toxins,
	while bringing about reactionary adaptations in consumers--\textbf{coevolutioon}.
	\item \textbf{Parasite-host interactions}: parasite benefits, while the host does not (+/-). Can cause
	rapid and dramatic changes in community dynamics by systematically wiping out entire chains of
	interspecific interactions (e.g., attacking chestnut trees, upon which a great number oof populations
	rely for survival).
\end{itemize}

\subsection{Interspecific competition}

The \textbf{ecological niche} of a specific species describes the biotic and abiotic resources that a species
consumes to survive. When the niches of two species overlap---that is, both species require access to the same
resource---, interspecific competition ooccurs, resulting in a decrease in carrying capacity.

The effects of interspecific competition on the carrying capacity of distinct populations has been proven in
experiments conducted by Russian ecologist G.F. Gause: three populations of similar ciliates grew to a smaller
popullation size in each oother's presence than in distinct environments, thus, proving the importance of
interspecific competition.

\subsection{The food chain}

In a community, the flow of energy from autotrophic self-feeding
\textbf{producers} is termed the \textbf{trophic structure}. Each organism above
the self-feeding producers is a heterotrophic \emph{consumer}, and derives its
consumed energy indirectly or directly from the autotrophic producers.
\textbf{Primary consumers} or herbivores, for example, directly rely on
autotrophic producers for energy. \textbf{Secondary consumers}, by contrast,
rely on primary consumers rely on primary consumer prey for sustinence. Examples
of organisms found in each group of the food chain are as follows:

\begin{enumerate}
	\item \textbf{Producers}: Plants on land, phytoplankton in water
	\item \textbf{Primary consumers}: grasshoppers, insects, snails,
	vertebrates (e.g., grazing mammals and birds) on land, zooplankton in water
	\item \textbf{Secondary consumers}: small mammals (e.g., mice, birds, frogs,
	spiders, lions, wolves, large carnivores that eat grazers) on land, small
	fishes that eat zooplankton in water
	\item \textbf{Tertiary consumers}: snakes on land
	\item \textbf{Quaternary consumers}: hawks on land, killer whales in water
\end{enumerate}

Additionally, scavenging organisms may consume dead material produced by any one
of the aforementioned organisms, while decomposer organisms digest organic
molecules by secreting digestive enzymes.

Alternatively, a \textbf{food web} may be used to present a more nuanced view of
a trophic structure.

\subsection{Describing a community}

In order to address the makeup of a community, one must first determine
the species richness of the community---that is, the number of species
contained inside the community---as well as the frequency of such species.
Keeping in mind the dependent relationships that exist in the trophic structure
of any community, it is inevitable that different relative abundences of certain
species will bear downstream effects. Take, for example, any given tree species.
A low relative abundance of such a species will result in decreasingly common
outbreaks of disease within the population, due to the sparse distribution of
the members of the population. Furthermore, as a result of the tree species' low
relative abundance, consumers that rely on its products will be, in much the same
manner, sparse in distribution.

Empirically, one might look to the experiments of Robert Pain, wherein it was
hypothesized that the species richness of a community is dependent on the
existence of a \textbf{keystone species}. Take, for example, the \emph{Pisaster}
sea star: in Paine's experiments, the lack of this organism in a community
resulted in a significantly lower species richness, as its absence allowed for
the prevalence of a mussel that preyed on invertebrates and algae in the community.

Typically, communities found on agricultural plots will exhibit low species
diversity, but a high \textbf{relative abundance} of the desired species. This
pattern of distribution is typically referred to as a \textbf{monoculture}.

\section{Development of a community}

\subsection{Disturbance as a mode of ecological succession}

Events such as storms, fires, floods, droughts, or even human intervention can be
described as \textbf{disturbances}, and each aid in a process of ongoing
\textbf{ecological succession}. Ecological succession is defined as the colonization
or replacement of a community's inhabitants---the life forms that may or may not exist
in any given geographic community---, and can be classified in terms of what is is
\emph{replacing}, or succeeding. Thus, the two types of ecological succession may be
defined as such:

\begin{itemize}
	\item \emph{Primary} succession: ecological succession that takes place in an
	environment that previously bore little life (e.g., retreating glaciers).
  \item \emph{Secondary} succession: ecological succession that replaces an existing,
	populated community, while keeping the underlying soil intact (e.g., after fires
	or floods).
\end{itemize}

\subsection{Damages of invasive species}

If introduced in environments where they lack sufficient natural predators, or other
biotic limitations, a species may become \textbf{invasive}---that is, they will begin
to colonize and dominate any suitable habitats, outside of the intended point of
introduction. Every year, more than $100$ billion is lost to invasive species.
Some examples are:

\begin{itemize}
	\item The burmese python
	\item Kudzu
	\item Zebra mussels
\end{itemize}

\section{Ecosystems}

In addition to the various species, and their respective interactions at some
geographic location, one might wish to study the various chemical and energy sources
at play in such an \textbf{environment}. \textbf{Energy flow} and \textbf{chemical
cycling} addresses such a desire, by describing the patterns by which energy is lost
as heat, and transferred between actors in an environment. In every environment,
energy must have a source. In many environments, light energy provided by the sun acts
as such an initial source. Chemical cycling, by contrast, deals strictly with substances
contained within the environment itself. Carbon and nitrogen, for example, are cycled
between the community and the various abiotic components of the environment.

\subsection{Energy budgeting in an ecosystem}

At any point in time, an ecosystem will receive energy from an input source. Most
commonly, light energy received from the sun may act as such energy, but must first
be converted to chemical energy in a process known as \textbf{primary production}.
While the total production of biomass in an ecosystem can be measured in terms of
\textbf{gross primary production}, not all chemical energy produced by producers
in a community will be readily available to the remaining consumers. Such ``remaining
energy'' is referred to as an ecosystem's \textbf{net primary production}.

\subsection{Decay of energy through an ecosystem's trophic structures}

Between each level in the trophic structure of an ecosystem, energy is lost in the
form of cellular expenses and an inability to digest. An \textbf{energy pyramid}
may be utilized to track such expenses among each trophic level by visually
representing the percentage of incorporated chemical energy among each trophic
level's biomass. An energy pyramid representing the flow of energy from producers
to meat-eating human consumers, for example, would aid in illustrating the vast
expenses introduced by cattle, which act as an intermediary between corn and meat-
eaters themselves.

\subsection{Chemical Cycles}

In contrast to an ecosystem's energy flow, the chemicals produced in an ecosystems
are not perpetually replenished---they simply traverse several biotic and abiotic
reservoirs (\textbf{biogeochemical cycles}). Carbon, for example, is stored globally
in the atmosphere---that is, it storage is not localized to a single ecosystem---,
while Nitrogen may be stored both locally in soil and globally in the atmosphere.
In local reservoirs, producers, consumers, and decomposers are the central actors
of the chemical cycle. The role of each actor can be defined in such a manner:

\begin{itemize}
	\item \textbf{Producers}: Introduce chemicals to the biotic portion of the
	biogeochemical cycle by creating complex organic compounds
	\item \textbf{Consumers}: Indirectly consume the organic compounds produced by
	producers
	\item \textbf{Decomposers}: Break down complex organic compounds, and return
	them to their inorganic constituent chemicals
\end{itemize}

\subsection{The Carbon Cycle}

Due to its ubiquitous nature as the chemical of life, carbon can be found in
atmospheric, plant/animal biomass, fossil fuel, soil, sedimentary rock, and
ocean reservoirs. In ecosystems, carbon is traferred between the various
biotic and abiotic components through the three aforementioned actors.
First and foremost, producers first incorporate carbon into the food supply
through the construction of complex organic molecules based on incoming \chem{CO2}.

\subsection{The Phosphorous Cycle}

\subsection{The Nitrogen Cycle}

\end{document}
