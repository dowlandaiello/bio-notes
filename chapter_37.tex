\documentclass{article}

\usepackage[english]{babel}
\usepackage{microtype}
\usepackage{graphicx}
\usepackage{wrapfig}
\usepackage{enumitem}
\usepackage{fancyhdr}
\usepackage{amsmath}
\usepackage{chemformula}
\usepackage{index}
\usepackage{hyperref}
\usepackage[margin=1.0in]{geometry}
\usepackage{qtree}
\usepackage{float}
\usepackage{booktabs}
\usepackage{tabularx}
\usepackage{textcomp}

\begin{document}
\title{Summary: Community Structure and Dynamics}
\author{Dowland Aiello}
\date{May 28, 2020}

\maketitle
\tableofcontents
\fancyhf{}

\newpage

\section{Overview: The Community as a Hierarchy}

In order to better analyze the various actors involved in some process, one might find that grouping various
actors involved at a certain geographical location or in respect to some inquiry can provide utility and make
answering such an inquiry easier. In doing so, one must analyze the various \textbf{populations} or groups of
interacting organisms of the same species, and, on a larger scale, the biological \textbf{communities} or groups
of geographically proximate populations that one deems paramount to successfully addressing the matter of inquiry.

In addition to the above classifications, a researcher may choose to bear regards to the composition of a community
--the feeding relationships, species, and number of populations within the community--, as well as the interactions
between the populations, which each may differ in their importance and scope. Over longer periods of time, these
interactions may manifest themselves in the study of community dynamics, which deals with the analysis of patterns
in the populatiion makeup of a community.

\section{Interactions in a Community}

Among the members of a community, \textbf{interspecific interactions} can be classified according to their effects
on the two populations involved. Interspecific \emph{competition}, for example, is generally a mutually harmlful
act, whereas \textbf{mutualism} (e.g., dichotomy of pollinators and flowers) is, in most cases, mutually 
beneficial. A third class of interspecific relations, by contrast, bears benefit to only one of the involved
parties. Examples of such a relationship are:

\begin{itemize}
	\item \textbf{Predation}: predator benefits, while the prey does not (+/-). Encourages adaptation in the
	prey population.
	\item \textbf{Herbivory}: consumer benefits, while the plant or algae does not (+/-). Encourages
	development of defense mechanisms in plants (e.g., thorns and spines), as well as chemical defense toxins,
	while bringing about reactionary adaptations in consumers--\textbf{coevolutioon}.
	\item \textbf{Parasite-host interactions}: parasite benefits, while the host does not (+/-). Can cause
	rapid and dramatic changes in community dynamics by systematically wiping out entire chains of
	interspecific interactions (e.g., attacking chestnut trees, upon which a great number oof populations
	rely for survival).
\end{itemize}

\subsection{Interspecific competition}

The \textbf{ecological niche} of a specific species describes the biotic and abiotic resources that a species
consumes to survive. When the niches of two species overlap---that is, both species require access to the same
resource---, interspecific competition ooccurs, resulting in a decrease in carrying capacity.

The effects of interspecific competition on the carrying capacity of distinct populations has been proven in
experiments conducted by Russian ecologist G.F. Gause: three populations of similar ciliates grew to a smaller
popullation size in each oother's presence than in distinct environments, thus, proving the importance of
interspecific competition.

\subsection{The food chain}

In a community, the flow of energy from autotrophic self-feeding
\textbf{producers} is termed the \textbf{trophic structure}. Each organism above
the self-feeding producers is a heterotrophic \emph{consumer}, and derives its
consumed energy indirectly or directly from the autotrophic producers.
\textbf{Primary consumers} or herbivores, for example, directly rely on
autotrophic producers for energy. \textbf{Secondary consumers}, by contrast,
rely on primary consumers rely on primary consumer prey for sustinence. Examples
of organisms found in each group of the food chain are as follows:

\begin{enumerate}
	\item \textbf{Producers}: Plants on land, phytoplankton in water
	\item \textbf{Primary consumers}: grasshoppers, insects, snails,
	vertebrates (e.g., grazing mammals and birds) on land, zooplankton in water
	\item \textbf{Secondary consumers}: small mammals (e.g., mice, birds, frogs,
	spiders, lions, wolves, large carnivores that eat grazers) on land, small
	fishes that eat zooplankton in water
	\item \textbf{Tertiary consumers}: snakes on land
	\item \textbf{Quaternary consumers}: hawks on land, killer whales in water
\end{enumerate}

Additionally, scavenging organisms may consume dead material produced by any one
of the aforementioned organisms, while decomposer organisms digest organic
molecules by secreting digestive enzymes.

Alternatively, a \textbf{food web} may be used to present a more nuanced view of
a trophic structure.

\end{document}
