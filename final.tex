\documentclass[12pt]{article}

\usepackage[english]{babel}
\usepackage{microtype}
\usepackage{graphicx}
\usepackage{wrapfig}
\usepackage{enumitem}
\usepackage{fancyhdr}
\usepackage{amsmath}
\usepackage{chemformula}
\usepackage{index}
\usepackage{hyperref}
\usepackage[margin=1.0in]{geometry}
\usepackage{qtree}
\usepackage{float}
\usepackage{booktabs}
\usepackage{tabularx}
\usepackage{textcomp}

\begin{document}
\title{On the Prohibition of Human and Non-Human Gene Patenting}
\author{Dowland Aiello}
\date{June 17, 2020}

\maketitle
\fancyhf{}

Even when accounting for inflation, healthcare spending per capita increased by
nearly 510\% between 1970 and 2018 \cite{healthcare_spending}. Furthermore,
according to data published in the Health Affairs journal, the percentage of
household income devoted to healthcare-related expenditures increased to 30\% in
2016 \cite{healthaffairs}. Thus, it has become apparent that the matter of
falling coverage rates, as well as rising premiums must be addressed. In doing
so, one might turn to the practice of \textbf{patenting} in the healthcare
sector, specifically in fields presenting a high cost of entry.
Pharmaceuticals, for example, have been notorious for their fiscally-intensive
research and development (R\&D) requirements, and were estimated to incur costs
of around \$2.8 billion in 2013 \cite{pharmaceutical_randd_costs}. By contrast,
genomic and CRISPR-based medical solutions were found to cost as much as five 
times \emph{more} than their pharmaceutical counterparts to
develop \cite{genetic_vs_traditional}. Thus, the employment of medical patents
by pharmaceutical and biotech companies is necessary to foster continued
innovation in the healthcare industry. However, it is with consideration to
the rulings and opionions of the supreme court, as well as the claims of the
plaintifss in relevant cases of arbitration that the assertion may be safely
made that the patenting of \textbf{non-synthetic} genes, human and nonhuman,
should be prohibited on the basis of the practice's categorical illegitimacy,
as well the demonstrable harm done to humans as a result of its adverse
market consequences.

By definition, a patent describes an invented process, product, or
methodology \cite{patent_laws}. Patents on non-synthetic genes, by contrast, do
not describe \emph{invented} products, but \emph{discovered} realities of nature
---the specific location of genes on a strand of DNA. As such, their existence
is categorically and inherently illegitimate, as it contradicts the legal
definition of a patent. In 2012, this definition was upheld by both the
Southern District Court of New York and the Supreme Court in
\emph{Association for Molecular Pathology v. Myriad Genetics, Inc.} on the
basis that such patents, "covered products of nature," and that, "laws of
nature, natural phenomena, and abstract ideas 'are basic tools of scientific
and technological work' that lie beyond the domain of patent protection,"
\cite{ass_v_myriad} thus explicitly reinforcing the aforementioned premise.
However, regardless of the illegitimacy of the practice of issuing patents
dealing explicitly with isolated gene segments, the isolation of specific
genes can still provide great utility, and may serve as the foundation of
patentable inventions (e.g., processes utilizing such genes). As such,
incentivization programs should be provided to maximize returns collectable
by companies engaging in this manner of research (e.g., grants, low-interest
loan contracts, and stimulus programs). Patents, however, may not be one such
program, as they necessitate the relegation of exclusive rights to the claimant
.

\begin{thebibliography}{7}

\bibitem{healthcare_spending}
	Kamal, Rabah. “How Has U.S. Spending on Healthcare Changed over Time?” Peterson-KFF Health System Tracker, Kaiser Family Foundation, 15 June 2020, \href{www.healthsystemtracker.org/chart-collection/u-s-spending-healthcare-changed-time/\#item-nhe-trends_year-over-year-growth-in-health-services-spending-by-quarter-2010-2019}{www.healthsystemtracker.org/chart-collection/u-s-spending-healthcare-changed-time/\#item-nhe-trends\_year-over-year-growth-in-health-services-spending-by-quarter-2010-2019}.

\bibitem{healthaffairs}
	McCarthy-Alfano, Megan. “Measuring The Burden Of Health Care Costs For Working Families.” Measuring The Burden Of Health Care Costs For Working Families | Health Affairs, HealthAffairs, 2 Apr. 2019, \href{www.healthaffairs.org/do/10.1377/hblog20190327.999531/full/}{www.healthaffairs.org/do/10.1377/hblog20190327.999531/full/}.

\bibitem{utilitypatents}
	Mayfield, Denise L. “Medical Patents and How New Instruments or Medications Might Be Patented.”
Missouri Medicine, vol. 113, no. 6, Nov. 2016, pp. 456–462., pmid:30228529.

\bibitem{genetic_vs_traditional}
	Irvine, Alison. “Paying for CRISPR Cures: The Economics of Genetic Therapies.” Innovative Genomics Institute (IGI), Innovative Genomics Institute, 18 Dec. 2019, \href{innovativegenomics.org/blog/paying-for-crispr-cures/}{innovativegenomics.org/blog/paying-for-crispr-cures/}.

\bibitem{pharmaceutical_randd_costs}
	Dimasi, Joseph A., et al. “Innovation in the Pharmaceutical Industry: New Estimates of R\&D Costs.” Journal of Health Economics, vol. 47, 12 Feb. 2016, pp. 20–33., doi:10.1016/j.jhealeco.2016.01.012.

\bibitem{patent_laws}
	“Patent.” Legal Information Institute, Cornell Law School, \href{www.law.cornell.edu/wex/patent}{www.law.cornell.edu/wex/patent}.

\bibitem{ass_v_myriad}
	Supreme Court of the United States. Association for Molecular Pathology v. Myriad Genetics, Inc. Vol. 569, 13 June 2013.

\end{thebibliography}

\end{document}
